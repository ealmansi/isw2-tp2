
\section{Arquitectura}

\subsection{Repositorios}

Los repositorios forman parte de un esquema de almacenamiento de datos distribu'ido. Un repositorio distribu'ido en un datacenter contiene informaci'on sobre la regi'on en la que se encuentra el datacenter.

\begin{itemize}
	\item \textbf{Apuestas por partido.} Contiene las apuestas involucradas en cada partido en ejecuci'on.
	\item \textbf{Partidos pendientes y en ejecucion.} Contiene los partidos (tanto de desaf'io simulaci'on como fantas'ia) que est'an pendientes y en ejecuci'on.
	\item \textbf{Servidores simulando o transmitiendo partido.} Contiene los servidores que est'an simulando o transmitiendo cada partido.
	\item \textbf{Estad'isticas de jugador.} Contiene informaci'on estad'istica, como por ejemplo la cantidad de partidos ganados y perdidos, de cada jugador.
	\item \textbf{Informaci'on de eventos reales.} Contiene toda la informaci'on sobre eventos reales en los que se basan partidos fantas'ia.
	\item \textbf{Usuarios conectados a desaf'io.} Contiene ID e IP de los usuarios actualmente conectados a la sala de un desaf'io.
	\item \textbf{Regiones habilitadas para transmisi'on de partidos fantas'ia.} Contiene las regiones a las que se les puede transmitir un partido real desde 'esta regi'on.
	\item \textbf{Datos de jugador.} Contiene los datos de un jugador de la regi'on.
	\item \textbf{Estado de partido en ejecuci'on.} Contiene los datos de estado, por ejemplo puntaje de cada equipo, de los partidos que se est'an ejecutando en este momento.
	\item \textbf{Fixtures de desaf'ios.} Contiene los fixtures actualizados de los desaf'ios que a'un no acabaron. En particular, contiene los fixtures de los desaf'ios que a'un no han empezado.
	\item \textbf{Puntaje por acciones.} Contiene informaci'on sobre los par'ametros que se utilizan para puntuar las acciones que ocurren en un partido fantas'ia.
	\item \textbf{Premios por desaf'io.} Contiene los premios (monetarios y no monetarios) que se otorgan a los participantes desaf'ios que a'un no han acabado.
	\item \textbf{M'etodos de pago.} Contiene la informaci'on sobre los servicios de pago que se aceptan en 'esta regi'on.
	\item \textbf{Informaci'on de redes sociales.} Contiene la informaci'on m'as reciente descargada de redes sociales.
	\item \textbf{Publicidades.} Contiene las publicidades m'as recientes recibidas desde los proveedores de publicidad.
\end{itemize}

\subsection{Diagrama}
\subsection{Explicación}
\subsubsection{General}
Lorem ipsum dolor sit amet, consectetur adipisicing elit, sed do eiusmod
tempor incididunt ut labore et dolore magna aliqua. Ut enim ad minim veniam,
quis nostrud exercitation ullamco laboris nisi ut aliquip ex ea commodo
consequat. Duis aute irure dolor in reprehenderit in voluptate velit esse
cillum dolore eu fugiat nulla pariatur. Excepteur sint occaecat cupidatat non
proident, sunt in culpa qui officia deserunt mollit anim id est laborum.


\subsubsection{Conectores auxiliares}
\textbf{Holy Connector:}
Este conector es un conector seguro, donde se tiene de cada extremo de la comunicacion un componente \textbf{comunicador} y entre ellos se comunican por un canal NO SEGURO. 
Por lo tanto de ambos lados nos vemos obligados a poner un componente de checksum, para conseguir la integridad de la informacion, un componente de encriptacion y desencriptacion, para poder lograr la confidencialidad y un componente que pantenemos el \textbf{estado de la comunicacion} con el cual ....


\textbf{Video Cuca Connector y Data Cuca Connector:} 
Ambos conectores funcionan de la misma manera, la unica diferencia es que uno es para streaming de datos y otro para streaming de videos. 
Contamos en ambos extremos con comunicadores de streaming que se comunican por conector Holy Connector (seguro) y tambien, en ambos lados tenemos compresores, para que en la transimicion no se transmita el streaming en si, sino una version comprimida.
En ambos extremos tenemos pipe como conectores de entrada a los componentes, esto es para hacer un \textbf{buffer} y cada tanto mandar muchos datos juntos, de esta forma evitamos los tipicos \textbf{buffering...} en los streaming.

\subsubsection{Atributos de Calidad vs Arquitectura}
Lorem ipsum dolor sit amet, consectetur adipisicing elit, sed do eiusmod
tempor incididunt ut labore et dolore magna aliqua. Ut enim ad minim veniam,
quis nostrud exercitation ullamco laboris nisi ut aliquip ex ea commodo
consequat. Duis aute irure dolor in reprehenderit in voluptate velit esse
cillum dolore eu fugiat nulla pariatur. Excepteur sint occaecat cupidatat non
proident, sunt in culpa qui officia deserunt mollit anim id est laborum.

