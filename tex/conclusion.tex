\section{Conclusi'on}

Mirando atr'as podemos ver que la parte de planificaci'on y definici'on de casos de uso nos result'o m'as simple que la de diseño de
la arquitectura. Creemos que 'esto se debe en gran parte a que el proceso utilizado en la primera parte sigue unas pautas m'as simples
ya que la complejidad de la tarea permite definirlas de una forma m'as estructurada.

El diseño de la arquitectura fue dif'icil, requiri'o discusiones largas del equipo entero y la evaluaci'on de los tradeoffs que implicaba
cada decisi'on constantemente. En varias situaciones barajamos alternativas que no presentaban un orden de calidad claro.
Realizar 'este trabajo nos brind'o la oportunidad de poder apreciar de primera mano la importancia de varios principios vistos en clase:
\begin{itemize}
\item La principales decisiones de arquitect'onicas deben ser tomadas por un pequeno grupo de arquitectos para conseguir que la
  arquitectura presente integridad conceptual. Nuestro equipo const'o de 4 personas y en abundantes ocasiones debimos charlar s'olo para
  alinear las visiones de cada uno. Si hubi'eramos repartido el diseño la calidad del sistema se hubiera visto perjudicada.
\item El valor de utilizar herramientas como el QAW. Identificar atributos de calidad result'o ser una tarea m'as compleja de lo que nos
  imagin'abamos. Sin el espacio que aport'o el QAW para que pudi'eramos encontrarnos con las necesidades de los stakeholders nos hubi'eramos
  perdido intentando suponer qu'e atributos de calidad eran prioritarios.
\item Enmarcar el proceso como un aprendizaje incremental. La arquitectura final result'o ser el producto de varias iteraciones que refinaron
  y resignificaron a las anteriores. Por ejemplo, la introducci'on de un nuevo componente nos instaba a revisar el nivel de detalle con el que
  hab'iamos definido otro.
\end{itemize}
Finalmente, un desafio que encontramos fue la eleccion del mejor nivel de detalle para cada una de las partes de la arquitectura. Varias veces diagramamos
una parte y luego nos dimos cuenta que no era relevante dentro de la arquitectura general.
\end{document}
