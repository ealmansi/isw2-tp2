\section{Arquitectura}
\subsection{Conectores propios}
\subsubsection{Notaci'on}

\begin{itemize}
\item Pipe \includegraphics[height=0.6cm]{diagramas/NPIPE} 
\item Holy Connector \includegraphics[height=0.5cm]{diagramas/NHC} 
\item Asynchronous Holy Connector \includegraphics[height=0.5cm]{diagramas/NHCCA}
\item Holy Video Connector \includegraphics[height=0.5cm]{diagramas/NHVC} 
\item Holy Data Connector \includegraphics[height=0.5cm]{diagramas/NHDC} 

\end{itemize}

\subsubsection{Holy Connector}

\begin{center}
	\includegraphics[height=7cm]{diagramas/HC} 
\end{center}

Este conector provee:

\begin{itemize}
	\item \textbf{Seguridad.} Mediante un componente de encriptaci'on y desencriptaci'on de los datos enviados.
	\item \textbf{Integridad.} Mediante un componente de checksum.
	\item \textbf{Confiabilidad de la conexi'on.} Mediante un componente que mantiene el estado de la conexi'on. Por ejemplo, este componente podr'ia implementar el protocolo TCP.
\end{itemize}

Observemos que entre los extremos utilizamos un conector tipo client-server que, asumimos, que se extiende sobre un medio inseguro. Por lo tanto, podemos pensar al Holy Connector como un cliente-server seguro, sobre un medio inseguro.

\subsubsection{Asynchronous Holy Connector}

\includegraphics[height=7cm]{diagramas/HCCA} 

Id'entico al Holy Connector, con la diferencia de que en lugar de usar un client-server como conector intermedio, utilizamos un conector de call asincr'onico.

\subsubsection{Holy Video Connector}

\includegraphics[height=4cm]{diagramas/HVC} 

Contamos en ambos extremos con comunicadores de streaming, que se comunican mediante un Asynchronous Holy Connector. Adicionalmente tenemos, en ambos extremos, compresores adecuados, para que la tasa de informaci'on transmitida por unidad de tiempo sea m'as alta. As'i, se puede transmitir video de m'as alta calidad, en tiempo razonable.

Un detalle importante es que haremos un peque\~no abuso de la funcionalidad del Asynchronous Holy Connector, asumiendo que al utilizarlo como conector aqu'i, en el Holy Video Connector, no se utiliza el componente de checksum. Hacemos 'esto para evitar definir un componente id'entico al Asynchronous Holy Connector pero sin el componente de checksum. La raz'on por la que no queremos usar 'este componente, es que no nos interesan chequeos de integridad, a lo largo de la comunicaci'on del flujo de datos. Por un lado, este chequeo demanda tiempo, que es escaso en el contexto de streaming. Por otro lado, no es necesario garantizar completa integridad, y se admiten peque\~nas diferencias entre los datos enviados y los recibidos.

En ambos extremos tenemos, adem'as, pipes como conectores de entrada y salida. Este buffering permite el env'io y recepci'on de datos flu'ida (evita los tipicos mensajes ``\textit{buffering...}'').


\subsubsection{Holy Data Connector}

\includegraphics[height=4cm]{diagramas/HDC} 

An'alogo al Holy Video Connector, pero con compresores adecuados para texto en lugar de video. Al igual que antes, asumimos que no se utiliza el componente de checksum en el conector que vincula los extremos.

\subsection{Diagrama}

\subsubsection{Arquitectura orientada a datacenters}

Nuestro sistema tiene una arquitectura distribuida, basada en datacenters. Un datacenter contiene una gran cantidad de servidores que proveen el servicio de juego. Dentro de un datacenter, los servidores se dividen en conjuntos, y cada conjunto provee servicio a una regi'on distinta. Una regi'on es uno o m'as pa'ises. En el caso extremo, una regi'on son todos los pa'ises del mundo. Notar que esto significa que un datacenter puede proveer servicio a varias regiones.

Todas las peticiones que ingresan en un datacenter son recibidas y procesadas por una m'aquina de tipo \textit{router}. Un router redirige la petici'on a alg'un servidor de la regi'on a la que est'e dirigida la petici'on. Los routers pueden rutear al conjunto de servidores de cualquiera de las regiones servidas en el datacenter. Por ejemplo, podr'ia haber un datacenter en Argentina, que provea el servicio para Argentina, Brasil y Uruguay; en este caso, los routers del datacenter procesan peticiones de las tres regiones.

El siguiente diagrama muestra lo descripto. 

\includegraphics[width=15cm]{diagramas/datacenter.png}

\noindent
El datacenter de la figura tiene dos routers, que procesan pedidos para dos regiones. Cada regi'on se compone de tres servidores. Notar que una de las regiones contiene dos tipos de repositorios, A1 y A2. Estos repositorios aparecen en todos los servidores de la regi'on, debido a que los datos est'an distribuidos sobre todos estos servidores. An'alogamente, los servidores de la otra regi'on, tienen repositorios B1 y B2, distribuidos. Cada servidor tiene un \textit{motor de bases de datos distribuidas}, que administra el acceso a los repositorios distribu'idos.

Este esquema de distribuci'on de datos sobre servidores es lo que permite que el sistema sea escalable. Adem'as provee otras bondades como capacidad de replicaci'on de datos, necesario para tener alta disponibilidad. M'as a'un, es posible implementar distribuci'on de datos entre datacenters, si hacemos que el conjunto de servidores de una regi'on est'e distribu'ido en m'as de un datacenter. Esto se muestra en la siguiente figura.

\includegraphics[width=15cm]{diagramas/datacenterx2.png}

\noindent
En particular, de esta forma podemos implementar replicaci'on de datos entre datacenters, y as'i si un datacenter se cae, no perdemos datos ni capacidad de proveer servicio a una regi'on.

\subsubsection{Descripci'on general}

Para describir la arquitectura, optamos por explicar c'omo se satisface cada una de los requerimientos indicados en el enunciado. Para 'esto, decidimos citar textualmente, cada p'arrafo del enunciado, y describir los aspectos del sistema relacionados. Si bien la descripci'on no abarca todos los aspectos del sistema (una explicaci'on exhaustiva demandar'ia demasiado tiempo y espacio), es un buen punto de partida para entender el funcionamiento general del mismo.

\textbf{\textit{Lo primero que se pretende es que abarque varios de los deportes colectivos más populares del planeta. Además del ya citado básquet, deberá incorporar a las ligas de fútbol, hockey, rugby, béisbol y fútbol americano que se determine tengan mayor número de seguidores (ej. el fútbol español o la NFL estadounidense), y que se consigan empresas que provean datos en tiempo real de la evolución del partido (tanto del desempeño de los jugadores como del partido en general). Esto último es importante porque además de la simulación de los mismos, ahora también se quiere incorporar un modo de liga de fantasía tradicional.}}

Para agregar deportes, lo 'unico que se debe hacer es contratar nuevos servicios de transmisi'on de partidos, agregar nuevos par'ametros del juego (i. e. agregar datos al repositorio de \textit{puntajes por acci'on}), y agregar funcionalidad al \textit{administrador de desaf'ios} para que se puedan crear desaf'ios de 'este nuevo deporte.

Los datos de la evoluci'on del partido provienen de los mismos servicios que nos transmiten partidos reales. Tanto los datos de un partido, como la transmisi'on del mismo nos son transmitidos en conjunto. Ver componente \textit{Servicio de transmisi'on de partido real}.

\textbf{\textit{En este nuevo modo los ganadores de los desafíos ya no se deciden a través de la simulación directa de los partidos, sino del desempeño de los jugadores en partidos reales en las ligas en cuestión. Por ej. usando el caso del básquet, se podría otorgar 1 punto por cada punto que haga ese jugador en el partido, 1.5 puntos por cada asistencia, 2 por cada bloqueo o robo, -1 por cada pérdida de balón, etc... El equipo que tenga mayor suma, gana el desafío.}}

El puntaje que se asigna a cada acci'on de cada deporte es un par'ametro que se encuentra en el repositorio \textit{puntajes por acci'on}.

\textbf{\textit{Con respecto a los desafíos, ahora pueden incluir un gran número de partidos, no sólo uno. En el caso de este nuevo modo ``liga fantasía tradicional'', podrían incluirse todos los partidos de una fecha determinada de una liga, o un salteado de un conjunto de fechas cualesquiera, o incluso la liga o torneo completo.
En el caso de la simulación, se arman torneos de diferentes formatos (se elige al momento de plantearse el desafío, playoff, liga, combinado zonas, etc.) con todos los equipos participantes. En este último caso se agrega la duración total “artificial” del torneo en semanas, pues se pretende que los partidos no se resuelvan más de forma inmediata si no que puedan “vivirse” como los
reales (más información luego).
}}

Los desaf'ios son creados por un cliente (un jugador o un administrador) desde la pantalla de lista de desaf'ios, mediante el m'odulo \textit{creador de desaf'ios}. 'Este se comunica con un \textit{administrador de desaf'ios}, del lado del servidor, el cual, mediante el \textit{administrador de fixtures}, crea el fixture correspondiente al desaf'io. El fixture puede tener diversas formas, dependiendo del tipo de desaf'io creado, por ejemeplo una liga, un torneo o un partido individual. El fixture es escrito en el repositorio \textit{fixtures de desaf'ios}.

Para que los desaf'ios se puedan extender a lo largo del tiempo, y se puedan vivir como si fuesen reales, existe la posibilidad de que los partidos de un desaf'io comiencen en distintos momentos. Esta informaci'on se incluye en el fixture.

Como los partidos comienzan en distintos momentos, el sistema contiene un m'odulo \textit{dispatcher de partidos simulaci'on} que sensa regularmente los fixtures, buscando los pr'oximos partidos a ejecutarse. Cuando llega la hora de ejecutar un partido, le informa al \textit{simulador de partido} que debe comenzar a ejecutar una simulaci'on. Le transmite la informaci'on de los equipos y los datos de los usuarios involucrados. Cuando el simulador concluye la ejecuci'on, le informa el resultado al \textit{procesador de resultados}.

Por un lado, este m'odulo le comunica al \textit{administrador de pagos} el ganador y el perdedor del partido, para que 'este efect'ue el pago correspondiente, seg'un las apuestas. Las apuestas se encuentran en el repositorio \textit{apuestas por partido en ejecuci'on}. Con toda esta informaci'on, el administrador de pagos le indica al sistema de pagos externo, las transacciones a realizarse.

Por otro lado, al terminar la simulaci'on, se le informa al administrador de fixtures c'omo debe modificar el fixture. Otros repositorios tambi'en son modificados adecuadamente.

\textbf{\textit{No sólo los jugadores pueden crear desafíos, sino que administradores propios del sitio pueden crearlos. Queda claro, que ahora los desafíos pueden ser aceptados por miles de jugadores, que competirán cada uno con su equipo para salir vencedor.}}

Los desaf'ios pueden ser creados desde un programa cliente hecho espec'ificamente para administradores.

\textbf{\textit{
Cada desafío tendrá un chat general para intercambiar mensajes con otros participantes, además de las conversaciones que podrá tener con competidores amigos a modo de ``Instant Messenger'' a través de la plataforma.
}}

Se incluye la posibilidad de chatear en las pantallas de partido. Un \textit{receptor/enviador de mensajes de chat} se encarga de recibir mensajes de chat, distribuirlos a todos los usuarios conectados a una sala, y filtrar y analizar aquellos que se refieran a jugadores de alguno de los equipos del partido, para modificar sus estad'isticas en tiempo real. Las modificaciones de las estad'isticas de los deportistas son almacenadas en el repositorio \textit{estad'isticas de deportistas}.

\textbf{\textit{
La idea principal con mayor consenso en los inversores, es que las fichas de apuesta pasan a ser reemplazadas por dinero real. Todos los participantes deberán poder ingresar datos de una tarjeta de crédito o cuenta corriente en entidades bancarias de los países participantes para que el sistema pueda debitar o acreditar dinero de forma inmediata y ser registrados por el sistema de forma segura.
}}

Justo antes de comenzar un partido, ocurre una ronda de apuestas entre los jugadores involucrados. Las apuestas, solicitadas por un \textit{administrador de apuestas}, se crean en el cliente por medio de un \textit{creador de apuestas}. Recibidas las apuestas en el servidor, el administrador las almacena en el repositorio \textit{apuestas por partido en ejecuci'on}.

\textbf{\textit{
Si bien el juego sigue permitiendo jugar absolutamente gratis, existirán desafíos que tengan una cuota de entrada para participar y repartirán un monto de dinero en premios en base a una cantidad mínima de participantes que deben anotarse para que el mismo pueda realizarse. Los premios que se pagarán serán en relación al rango de posiciones en el que queda el equipo del jugador al terminar el desafío (ej: 1er puesto, 2do, 3ro, 4to a 10mo, 10 a 50, 50 a 200, 200 a 1000, etc.).
Los desafíos “gratis” pueden otorgar premios como créditos no retirables para jugar, productos exclusivos de “partners”, tickets para eventos deportivos, etc.
}}

La cuota de entrada y los premios no monetarios se fijan a la hora de crear un desaf'io, y son procesados por el administrador de desaf'ios. Esta informaci'on es almacenada en el repositorio \textit{premios por desaf'io}.

\textbf{\textit{Una vez que comienza el primer partido de un desafío ya no se pueden hacer cambios en el equipo registrado o salirse del mismo o ingresar nuevos participantes. Cada desafío deberá tener una cuenta regresiva para indicar el tiempo restante para el comienzo. En todo momento, incluso mientras los partidos se están desarrollando, se debe poder ver quiénes son los participantes del mismo y el puesto actual del equipo del participante en el desafío actualizado al minuto a minuto de los partidos.}}

Para inscribirse a un desaf'io, un usuario utiliza el \textit{listador de desaf'ios} que, adem'as de mostrar todos los desaf'ios disponibles, permite inscribirse a ellos. La informaci'on de la inscripci'on es enviada a un administrador de desaf'ios, que le ordena al administrador de fixtures modificar el fixture con el nuevo inscripto. Comenzado un desaf'io, el administrador de fixtures no aceptar'a ning'un cambio sobre el fixture. Por lo tanto, no habr'a nuevas inscripciones.

Asimismo es posible recuperar la informaci'on sobre los participantes de un desaf'io en curso, a trav'es del fixture del desaf'io. Los rankings de un desaf'io son confeccionados por un \textit{calculador de rankings de desaf'ios} que, bas'andose en un fixture, calculan la posici'on de cada jugador.

\textbf{\textit{
De hecho, el estado de los partidos de los desafíos (en cualquiera de sus modos) debería poder seguirse en vivo a través del sitio, como cualquier sitio de noticias deportivo. Además, se está muy avanzado en conversaciones para acceder a streams de videos de diferentes ligas y que el partido pueda seguirse legalmente a través de la plataforma en algunas regiones habilitadas. Los dueños de los derechos de transmisión, a cambio, quieren poder vender espacios de publicidad en el sitio y acceder a algunos datos de preferencias/comportamiento de los usuarios, además de sus cuentas de mails para efectuar campañas de marketing más específicas.
}}

Cuando un usuario ingresa a la sala de un desaf'io, se conecta a un servidor que est'e simulando o proveyendo una transmisi'on de partido real, y 'este comienza a transmitirle datos.

En el caso del modo fantas'ia, el permiso de acceso a una transmisi'on est'a almacenado en el repositorio \textit{regiones habilitadas para transmisi'on de partidos fantas'ia}. Cuando un usuario solicita acceso a la sala de un desaf'io, primero debe resolverse la ubicaci'on de un datacenter que contenga la informaci'on sobre ese desaf'io. Ubicado un tal datacenter, se verifica el permiso de acceso a la transmisi'on, y en caso que la regi'on del usuario est'e habilitada, se le concede acceso a un servidor que est'e proveyendo la transmisi'on.

A los due\~nos de los derechos de transmisi'on se les puede dar acceso a los repositorios de datos de usuarios f'acilmente, consultando cualquiera de los datacenters. A su vez, tenemos almacenadas publicidades que ser'an mostradas en pantallas de desaf'ios. Las publicidades son recibidas a trav'es de proveedores de publicidad externos, y almacenadas en un repositorio \textit{publicidades}.

\textbf{\textit{
Con respecto al modo de simulación, se pretende extender las reglas utilizando a un comité de expertos (jugadores, técnicos, estadísticos, periodistas deportivos, etc.) de cada deporte para intentar que se asemeje aún más a la realidad. En el caso del básquet incorporará fouls, tiros libres, cambios de jugadores (suplentes), minutos en cancha + cansancio, tiempos con cambios de lado de la cancha, estadios locales y visitantes, condiciones climáticas, movimiento de los jugadores y sus posiciones en la cancha al efectuar acciones, etc. para extrapolar más certeramente la actuación del jugador en base a sus estadísticas.
Al mismo tiempo se busca modificar el simulador para que en vez de un log de salida, provea un stream continuo de “minuto a minuto” mucho más detallado que el log de la versión anterior. La razón de esta decisión es que se está por llegar a un acuerdo por la compra de un motor gráfico que permita a los participantes ver el partido a partir del stream de salida del simulador. En este momento se negocia por al menos dos motores diferentes.
}}

Las reglas de los juegos simulados est'an completamente contenidas en el simulador. Su modificaci'on no afecta al funcionamiento del resto de nuestro sistema.

Un simulador de partido, ubicado en el servidor, env'ia un flujo de datos a todos los clientes conectados al mismo. Cuando un cliente recibe datos de la simulaci'on, son procesados por un motor gr'afico, el cual posteriormente los dibuja en la interfaz gr'afica.

\textbf{\textit{
El primero es un motor gráfico 3d de última generación que muestra a los jugadores con un gran nivel de realismo pero que puede ser muy pesado para la mayoría de los dispositivos con los que los participantes accederían al portal (que va desde PCs de escritorio a dispositivos móviles de todo tipo). La empresa desarrolladora habla de que se le podría diseñar un módulo para transmitir en video el partido a los dispositivos no soportados por el motor gráfico. Otra empresa presenta un motor 2d más humilde pero mucho menos demandante, donde los partidos se ven “desde arriba” con un diseño caricaturezco y que funciona para dispositivos móviles con varios años de antigu\"uedad. Se cree que se va a llegar a un acuerdo con ambas compañías para incorporar ambos motores en la primera versión, según se necesite.
}}

Se decidi'o incluir ambos motores gr'aficos en el programa cliente. Por lo tanto, cuando se detecte que el dispositivo cliente no tiene prestaciones suficientemente buenas para utilizar el motor 3D, se utilizar'a el motor 2D. No se enviar'an datos de video 3D preprocesados desde el servidor.

\textbf{\textit{
Cualquier participante, incluso si no forma parte del desafío debería poder ver un partido simulado de su región. Al ser ahora “visibles” las simulaciones, las mismas dejan de ser instantáneas, y se intenta que las duraciones sean similares a las de un partido normal. De esta forma se puede vender publicidad específica a través del sitio. También se quiere aumentar el caudal de redes sociales que se utilizan para “afectar” las estadísticas (no sólo Twitter, sino incorporar Facebook, Google+, etc.) y que incluso los comentarios de los usuarios en los chats de los desafíos o incluso los IMs (Instant Messages) privados de los usuarios puedan ser utilizados para afectar los resultados. Como las simulaciones ahora duran similarmente a un partido real, los resultados de menciones pueden verse como un “rating minuto a minuto” de cada jugador.
}}

Un usuario puede solicitar ver cualquier partido de la lista de desaf'ios y, si su regi'on est'a habilitada, un servidor adecuado se lo transmitir'a. La duraci'on del partido estar'a dado por el flujo de datos transmitido del servidor al cliente.

Las redes sociales son constantemente sensadas por un \textit{administrador de redes sociales}, que se encarga de extraer datos relevantes relacionados con deportes. Estos datos son almacenados en un repositorio \textit{informaci'on de redes sociales}. Esta informaci'on es sensada peri'odicamente por un \textit{analizador de informaci'on de redes sociales}, el cual procesa la informaci'on y se la transmite al simulador de partido. Adem'as, actualiza el repositorio de estad'isticas de jugadores.

\textbf{\textit{Se espera que el sistema sea utilizado por millones de personas a lo largo y ancho del planeta y que se llegue a ese número de suscriptos muy rápidamente (hay mucho interés de que uno de los países incluidos en la versión inicial sea China). Algunos especialistas en redes plantearon entonces la necesidad de regionalizar la plataforma en varios niveles y que los participantes de cada región sólo compitan entre sí en desafíos locales o nacionales (que puede ser de las ligas del mundo más seguidas para esa región). Además, se cree, esto debería facilitar la logística con los bancos, el idioma de los chats, los pagos en la moneda local, la entrega de premios, etc.
}}

Como vimos, la arquitectura del sistema est'a orientada a datacenters. Si se quisiera habilitar el servicio en China, una buena idea ser'ia construir un datacenter en el continente asi'atico. Dicho datacenter contendr'ia servidores que sirven, en particular, al antedicho pa'is.

Ya hemos hablado de la organizaci'on de servidores por regiones. Para tener ligas locales, continentales y mundiales, podemos crear regiones que abarquen los pa'ises que deseamos que conformen la liga. Para nuestro sistema, cualquier conjunto de pa'ises es id'entico, con lo cual podemos servir un campeonato mundial de la misma forma que servimos un torneo local. El 'unico detalle es que en el caso de una liga local, lo m'as razonable es ubicarla en un datacenter cercano al pa'is en el que se desarrolla la liga, mientras que en una liga mundial, la mejor elecci'on del (o los) datacenter que proveer'a el servicio no es tan obvia. Puede que cualquier datacenter sea una buena elecci'on.

\textbf{\textit{El ranking de los jugadores más ganadores del prototipo se mantiene pero con algunos cambios, ya que desafíos con muchos usuarios o muy largos en tiempo otorgarán mayor puntaje que los “mano a mano” convencionales. Dicho ranking servirá para acceder a torneos de niveles regionales o internacionales. Por ejemplo, en base a las limitaciones del hardware y conexiones con el que se cuenta para servidores, los mejores 10.000 participantes de Argentina podrán acceder a una nueva lista de desafíos con los restantes 10.000 mejores jugadores de cada país del resto de Latinoamérica. Y los mejores 1.000 de esta nueva región podrán competir en un nivel mundial con los 1.000 mejores de Europa, Asia, Norteamérica, África y Oceanía.
Para un participante mantenerse en los rankings de cada nivel, los resultados de la región mundial, no influirán en los del nivel regional (por ej. Europa), ni en los de nivel nacional o local, aunque dejar de competir por un tiempo en alguno hará restar posiciones, lo que eventualmente puede significar perder acceso a los niveles superiores. Se quiere tener campeones mundiales anuales, y que las finales sean verdaderos eventos a transmitirse en vivo a los millones de usuarios del sitio de todas las regiones para luego hacer promociones y giras con ellos alrededor del mundo. Esto implica que los administradores pueden hacer visibles desafíos de niveles superiores a usuarios que no tienen normalmente acceso para que el resto de los usuarios puedan seguirlos en vivo. Se plantea que cada vez que comience una nueva temporada o año, se reinicie el ranking del sitio.
}}

Cuando un administrador crea un desaf'io internacional, se computa el ranking de todos los jugadores de las regiones intervinientes en el desaf'io. 'Esto lo hace un \textit{calculador de rankings de usuarios}. El resultado se almacena en el repositorio \textit{rankings de usuarios}.

Con esta informaci'on, cada vez que un usuario desea inscribirse a dicho desaf'io, el administrador de desaf'ios consulta el ranking, y verifica si el usuario se encuentra entre los primeros $n$ del ranking.

Observar que los resultados de la regi'on mundial no afectan a las estad'isticas de otras regiones, gracias a que los servidores de partido s'olo modifican las estad'isticas de usuario de su conjunto. En otras palabras, los servidores de regi'on mundial s'olo modifican la base de datos distribu'ida de regi'on mundial.

\textbf{\textit{
Es importante que no sólo los participantes puedan acceder a su estado de cuenta, sino que administradores del más alto nivel de privilegios también puedan hacerlo para cualquier jugador. Además, deberían ser capaces de ver los movimientos y el balance del sitio de juegos en cada región o a nivel global. Como se quiere evitar potenciales problemas con entidades gubernamentales de control (fundamentalmente del fisco) en los países que opere el sitio, se desea que estén diseñados e implementen mecanismos de tal manera que se pueda brindar la transparencia deseada, tanto en movimientos de dinero, como en el funcionamiento correcto de las simulaciones.
}}

Todos los programas cliente de tipo administrador cuentan con un \textit{administrador de cuentas de usuario}, que permite visualizar toda la informaci'on disponible sobre los usuarios.

Se garantiza la seguridad de las transacciones utilizando conectores Holy, que garantizan seguridad. Todas las transacciones pueden ser listadas tanto por usuarios como administradores, a trav'es del \textit{listador de transacciones}, el cual consulta el repositorio de \textit{transacciones}.

\textbf{\textit{Por último, se sabe que la legislación de varios países prohíbe el acceso a este tipo de juegos. Dependiendo del caso, se deberá evitar que: directamente el usuario acceda al sitio o que no puedan registrarse o bien registrarse, pero sólo jugar los desafíos gratuitos sin poder ganar premios o pudiendo ganar premios que no sean sumas de dinero.}}

Cuando un usuario solicita registrarse en el sistema, un componente \textit{registrador} verifica que el usuario pertenezca a una regi'on habilitada para jugar. Dicha informaci'on se encuentra en el repositorio \textit{regiones habilitadas para jugar/apostar}. An'alogamente, cuando un usuario solicita inscribirse a un desaf'io, el administrador de desaf'ios chequea contra mismo el repositorio, que el usuario efectivamente tenga permiso para apostar.

\subsection{Repositorios distribuidos}

A continuaci'on listamos la informaci'on contenida en cada uno de los repositorios. Recordemos que todos ellos forman parte de un esquema de almacenamiento de datos distribu'ido. Conocer exactamente qu'e informaci'on tiene a disposici'on el sistema, es 'util para comprender sus capacidades.

\begin{itemize}
	\item \textbf{Ranking de usuarios.} Contiene el ranking de los usuarios de la regi'on.
	\item \textbf{Apuestas por partido en ejecuci'on.} Contiene las apuestas involucradas en cada partido en ejecuci'on.
	\item \textbf{Servidores simulando o transmitiendo partido.} Contiene los servidores que est'an simulando o transmitiendo cada partido.
	\item \textbf{Partidos pendientes y en juego.} Contiene los partidos (tanto de desaf'io simulaci'on como fantas'ia) que est'an pendientes y en ejecuci'on.
	\item \textbf{Regiones habilitadas para jugar/apostar.} Contiene las regiones a las que se les puede prestar servicio.
	\item \textbf{Estad'isticas de usuario.} Contiene informaci'on estad'istica, como por ejemplo la cantidad de partidos ganados y perdidos, de cada jugador.
	\item \textbf{Informaci'on de eventos reales.} Contiene toda la informaci'on sobre eventos reales en los que se basan partidos fantas'ia.
	\item \textbf{Usuarios conectados a partido.} Contiene ID e IP de los usuarios actualmente conectados a la sala de un desaf'io.
	\item \textbf{Regiones habilitadas para transmisi'on de partidos fantas'ia.} Contiene las regiones a las que se les puede transmitir un partido real desde 'esta regi'on.
	\item \textbf{Datos de cuenta de usuario.} Contiene los datos de un jugador de la regi'on.
	\item \textbf{Estado de partido en ejecuci'on.} Contiene los datos de estado, por ejemplo puntaje de cada equipo, de los partidos que se est'an ejecutando en este momento.
	\item \textbf{Fixtures de desaf'ios.} Contiene los fixtures actualizados de los desaf'ios que a'un no acabaron. En particular, contiene los fixtures de los desaf'ios que a'un no han empezado.
	\item \textbf{Puntaje por acciones.} Contiene informaci'on sobre los par'ametros que se utilizan para puntuar las acciones que ocurren en un partido fantas'ia.
	\item \textbf{M'etodos de pago.} Contiene la informaci'on sobre los servicios de pago que se aceptan en 'esta regi'on.
	\item \textbf{Informaci'on de redes sociales.} Contiene la informaci'on m'as reciente descargada de redes sociales.
	\item \textbf{Publicidades.} Contiene las publicidades m'as recientes recibidas desde los proveedores de publicidad.
	\item \textbf{Transacciones.} Contiene registro de todas las transacciones (debitos y acreditaciones) realizadas por el sistema.
	\item \textbf{Premios por desaf'io.} Contiene los premios (monetarios y no monetarios) que se otorgan a los participantes desaf'ios que a'un no han acabado.
	\item \textbf{Estad'isticas de deportista.} Contiene la informaci'on estad'istica de los deportistas que los simuladores usan durante su ejecuci'on.
	
\end{itemize}

\subsubsection{Atributos de Calidad vs Arquitectura}
A continuacion listamos todos los atributos de calidad detallados en la seccion anterior y explicamos como hicimos para resolverlo en el diagrama de componentes y conectores.

\begin{itemize}
\item \textbf{Atributo:} El sistema debe evitar que los servidores superen su l'imite de clientes y se caigan.

 \textbf{Justificaci'on:} Los componentes \textit{router} que reciben las peticiones de un cliente mantienen contadores acerca de la carga de los servidores. Cuando detectan una posible sobrecarga consultan la cantidad de usuarios conectados a los servidores de una regi'on. Lo hacen consultando el repositorio \textit{usuarios conectados a desaf'io}. Si hay sobrecarga, no se responde a las peticiones.

\item \textbf{Atributo:} Desea que haya un esfuerzo por respetar a los países en los que su legislación no permite que directamente se ingrese al sitio.

\textbf{Justificaci'on:} Los \textit{router} que reciban peticiones provenientes de regiones no habilitadas, no ser'an resueltas. Las regiones no habilitadas se consultan en el repositorio \textit{regiones habilitadas para jugar}.

\item \textbf{Atributo:} Se requiere que sea fácil desactivar una cuenta por un tiempo, para ayudar a los adictos en recuperación.

\textbf{Justificaci'on:} Se incluye un m'odulo para desactivar una cuenta, en el programa cliente.

\item \textbf{Atributo:} El simulador debe poder extenderse para soportar reglamentaciones nuevas ya que se espera poder incluir pa'ises adicionales.

 \textbf{Justificaci'on:} La modificabilidad el simulador corre por cuenta de los desarolladores del mismo. En el caso del modo fantas'ia, cuya operatoria depende de nuestros desarrolladores, se incluye un repositorio con datos de configuraci'on para los partidos de dicho modo.

\item \textbf{Atributo:} El sistema debe correr en la mayor cantidad de plataformas posible. El sistema debe poder extenderse f'acilmente para incluir nuevas plataformas.

 \textbf{Justificaci'on:} La arquitectura del sistema no asume requerimientos de hardware o software (salvo algunos muy b'asicos) sobre el cliente. Para poder soportar todo el abanico de dispositivos, desde aquellos con recursos de hardware y software limitados como aquellos de altas prestaciones, se incluyen dos motores gr'aficos, uno 3D (de altos requerimientos) y otro 2D (de bajos requerimientos). La decisi'on sobre cu'al utilizar se toma en base a los recursos disponibles al momento de la recepci'on de los resultados de la simulaci'on.

%\item Atributo: Se deben poder agregar nuevos criterios para hacer data mining f'acilmente.
%\item Justificaci'on:

%\item Atributo: Deben poder crearse nuevos desaf'ios en base a los datos minados, las estad'isticas disponibles de los usuarios.
%\item Justificaci'on:

\item \textbf{Atributo:} Se quiere poder controlar las publicidades en las simulaciones y el sitio en general en base a el tipo de audencia.

 \textbf{Justificaci'on:} El Motor de bases de datos distribuida tiene la capacidad de hacer un analisis de datos para poder saber, dado un usuario que publicidad mostrarle.

\item \textbf{Atributo:} La simulaci'on debe poder mejorarse para que sea m'as realista de manera incremental sin que los cambios sean costosos.

 \textbf{Justificaci'on:} Las mejoras de la simulacion asumimos que se hacen dentro del componente "Simulador de partido".

\item \textbf{Atributo:} Los datos de pago deben guardarse para que el usuario s'olo tenga que actualizarlos espor'adicamente.

 \textbf{Justificaci'on:} Contamos con un repo "Metodos de pago" donde guardamos los metodos de pago de cada usuario. Un usuario puede mediante el "Persistidor Metodo de Pago" guardarlos en ese repo.

\item \textbf{Atributo:} Se debe utilizar el engine 3D siempre que sea posible. Si el cliente no lo soporta adecuadamente debe utilizarse el 2D.

 \textbf{Justificaci'on:} Del lado del cliente contamos con un multiplexor, el cual, chequeando los requerimientos y funcionalidades del sistema cliente decidir'a que engine utilizar

\item \textbf{Atributo:} Los desf'ios globales que requiere el streaming a m'ultiples regiones simult'aneamente debe funcionar satisfactoriamente. La experiencia del usuario no debe verse afectada.

 \textbf{Justificaci'on:} Contamos con un conector de streaming que contiene un buffering para poder transmitir con mayor velocidad. Y por otro lado contamos con muchos servidores para que transmitan al mismo tiempo. 

\item \textbf{Atributo:} Los pagos y las credenciales de pago de los usuarios deben manejarse con seguridad, debe impedirse que hackers redireccionen los pagos o interfieran con las transacciones.

\textbf{Justificaci'on:} Para esto utilizamos el conector Holy, el cual es un conector seguro explicado en la seccion anterior, entonces la comunicacion con el Servidor de Pago es segura.

\item \textbf{Atributo:} Los datos de usuarios deben estar protegidos contra robos. Tiene que asegurarse la confidencialidad e integridad de los mismos.

 \textbf{Justificaci'on:} Para esto utilizamos el conector Holy, el cual es un conector seguro explicado en la seccion anterior, entonces la comunicacion con el Servidor de Pago es segura.

\item \textbf{Atributo:} Todas las transacciones de dinero deben estar logueadas de manera segura para poder presentarlas como evidencia a las autoridades de cada regi'on.

 \textbf{Justificaci'on:} Contamos con un repo de transacciones donde se loguean las transacciones. A la hora de realizar un pago el "Administrador de pagos" se comunica con el "Motor de bases de datos distribuidas" para que actualice el repo.

\end{itemize}

