\section{Conclusi'on}

En retrospectiva, el diseño de la arquitectura resultó mucho más desafiante que la planificación del proyecto y la identificación de los casos de uso relevantes. El desarrollo de las primeras instancias es relativamente estructurado, mientras que construir una arquitectura para el sistema completo brinda una flexibilidad mucho mayor, y es por lo tanto más compleja y se presta a la creatividad del diseñador y la evaluación de múltiples trade-offs. En algunos casos, esto conllevó debates extensos entre los integrantes del grupo con la finalidad de decidir entre sendas alternativas sin parámetros claros para compararlas en términos de calidad.

Distintos principios vistos en clase se manifestaron a lo largo de este trabajo, permitiéndonos apreciar en forma directa los siguientes puntos:

\begin{itemize}
\item El proceso de decisión. La principales decisiones arquitect'onicas deben ser tomadas por un pequeño grupo de arquitectos en forma centralizada para garantizar la integridad conceptual de la arquitectura desarrollada. Nuestro equipo const'o de cuatro personas elaborando una estrategia en conjunto y discutiendo en múltiples ocasiones para lograr alinear las visiones de cada integrante; de haber dividido el diseño y desarrollado cada parte independientemente, la calidad del sistema se hubiera visto perjudicada.
\item Identificación de atributos de calidad. Determinar el órden relativo de relevancia entre los atributos de calidad result'o ser una tarea compleja, remarcando el valor en el uso de herramientas como el QAW. Sin este espacio para encontrarnos con las necesidades de los stakeholders, hubiera sido muy difícil estimar cu'ales atributos de calidad eran prioritarios y, en consecuencia, se habría dificultado el criterio para discriminar entre las múltiples variantes presentadas en la etapa de diseño.
\item Enmarcar el proceso como un aprendizaje incremental. La arquitectura final result'o ser el producto de varias iteraciones que refinaron
  y resignificaron a las anteriores. Por ejemplo, la introducci'on de un nuevo componente nos instaba a revisar el nivel de detalle con el que
  hab'iamos definido otro.
\end{itemize}

Otro desaf'io incidental que se presentó fue a la hora de trasmitir la arquitectura desarrollada; en particular, la determinación del mejor nivel de detalle para cada una de las partes que componen la  arquitectura. Se intenta simult'aneamente ser completo y preciso en la descripci'on, pero de gran importancia es no perder claridad en la trasmisi'on de la informaci'on que se quiere compartir mediante el diagrama. En separadas ocasiones, se diagramaron partes con excesivo nivel de detalle, que luego consideramos no era relevante dentro de la arquitectura general y se dejaron de lado con la finalidad de obtener un diagrama más efectivo.

\end{document}
