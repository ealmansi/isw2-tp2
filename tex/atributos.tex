\section{Atributos de calidad}


\subsection{Disponibilidad}

\begin{center}
  \begin{tabular}{| l | p{10cm} | }
    \hline
  \textbf{Descripción} & Los partidos deben servirse con excelente calidad y sin cortes a pesar de que servidores de la empresa proveedora se pueden caer sin previo aviso.\\  \hline
  \textbf{Fuente} & Servidor del proveedor\\  \hline
  \textbf{Estímulo} & Se cae, por lo tanto deja de enviar datos.\\  \hline
  \textbf{Entorno} & Normal\\  \hline
  \textbf{Artefacto} & Sistema\\  \hline
  \textbf{Respuesta} & Se identifica la ca'ida del server por timeout y se inicia una nueva comunicaci'on con otro server.\\  \hline
  \textbf{Medición} & El tamano del buffer de streaming es suficiente para que s'olo se perciba la ca'ida 0,001 seg/hora dada que el uptime de los proveedores del servidor es de 95\%.\\  \hline
  \end{tabular}
\end{center} 

%% \begin{center}
%%   \begin{tabular}{| l | p{10cm} | }
%%     \hline
%%   \textbf{Descripción} & Los servidores de streaming de cada region deben ser acordes a la calidad de video usualmente transmitida a los usuarios de dicha region.\\  \hline
%%   \textbf{Fuente} & UNA FUENTE\\  \hline
%%   \textbf{Estímulo} & UN ESTIMULO.\\  \hline
%%   \textbf{Entorno} & UN ENTORNO\\  \hline
%%   \textbf{Artefacto} & UN ARTEFACTO\\  \hline
%%   \textbf{Respuesta} & UNA RESPUESTA\\  \hline
%%   \textbf{Medición} & UNA MEDICION\\  \hline
%%   \end{tabular}
%% \end{center} 

\begin{center}
  \begin{tabular}{| l | p{10cm} | }
    \hline
  \textbf{Descripción} & El sistema debe evitar que los servidores superen su l'imite de clientes y se caigan.\\  \hline
  \textbf{Fuente} & Usuario\\  \hline
  \textbf{Estímulo} & Se hace un pedido.\\  \hline
  \textbf{Entorno} & Sistema con servidor saturado\\  \hline
  \textbf{Artefacto} & Sistema\\  \hline
  \textbf{Respuesta} & Se selecciona un servidor que no est'a saturado y se direcciona al cliente a 'el.\\  \hline
  \textbf{Medición} & El 0.01\% de veces un servidor se satura y se direcciona un cliente a 'el antes de que la notificaci'on de que est'a saturado llegue al router.\\  \hline
  \end{tabular}
\end{center} 

\begin{center}
  \begin{tabular}{| l | p{10cm} | }
    \hline
  \textbf{Descripción} & Desea que haya un esfuerzo por respetar a los países en los que su legislación no permite que directamente se ingrese al sitio.\\  \hline
  \textbf{Fuente} & Externa\\  \hline
  \textbf{Estímulo} & Un usuario proveniente de un pais que en su legislacion no permite ingresar al sitio intenta ingresar.\\  \hline
  \textbf{Entorno} & Normal\\  \hline
  \textbf{Artefacto} & Sistema\\  \hline
  \textbf{Respuesta} & Se muestra una pantalla con una leyenda indicando que esta prohibido el ingreso.\\  \hline
  \textbf{Medición} & En el 0,001\% de los casos un usario perteneciente a un pais que prohibi el sitio logra ingresar.\\  \hline
  \end{tabular}
\end{center} 


\begin{center}
  \begin{tabular}{| l | p{10cm} | }
    \hline
  \textbf{Descripción} & Se requiere que sea fácil desactivar una cuenta por un tiempo, para ayudar a los adictos en recuperación.\\  \hline
  \textbf{Fuente} & Usuario adicto\\  \hline
  \textbf{Estímulo} & Un usuario adicto intenta conectarce.\\  \hline
  \textbf{Entorno} & Normal\\  \hline
  \textbf{Artefacto} & Sistema\\  \hline
  \textbf{Respuesta} & Se bloquea el acceso al sistema.\\  \hline
  \textbf{Medición} & El 99,99\% de los casos que un usuario adicto intenta conectarse desde una cuenta desactivada el acceso al sistema es bloqueado.\\  \hline
  \end{tabular}
\end{center} 





\subsection{Modificabilidad}

\begin{center}
  \begin{tabular}{| l | p{10cm} | }
    \hline
  \textbf{Descripción} & El simulador debe poder extenderse para soportar reglamentaciones nuevas ya que se espera poder incluir pa'ises adicionales.\\  \hline
  \textbf{Fuente} & Desarrollador\\  \hline
  \textbf{Estímulo} & Se quiere agregar reglamentaciones nuevas para incluir paises adicionales.\\  \hline
  \textbf{Entorno} & Normal\\  \hline
  \textbf{Artefacto} & Sistema en ejecucion\\  \hline
  \textbf{Respuesta} & Se realiza la extension sin problemas.\\  \hline
  \textbf{Medición} & Se modifican 2 componentes del sismtema.\\  \hline
  \end{tabular}
\end{center} 

\begin{center}
  \begin{tabular}{| l | p{10cm} | }
    \hline
  \textbf{Descripción} & El sistema debe correr en la mayor cantidad de plataformas posible. El sistema debe poder extenderse f'acilmente para incluir nueva.\\  \hline
  \textbf{Fuente} & Desarrollador\\  \hline
  \textbf{Estímulo} & Se quiere agregar nuevas plataformas donde correr el sistema.\\  \hline
  \textbf{Entorno} & Normal\\  \hline
  \textbf{Artefacto} & Sistema\\  \hline
  \textbf{Respuesta} & Se agrega con exito una nueva plataforma donde correr el sistema.\\  \hline
  \textbf{Medición} & Se agregan algoritmos en 3 componentes del sistema.\\  \hline
  \end{tabular}
\end{center} 

\begin{center}
  \begin{tabular}{| l | p{10cm} | }
    \hline
  \textbf{Descripción} & Se quiere poder controlar las publicidades en las simulaciones y el sitio en general en base a el tipo de audencia.\\  \hline
  \textbf{Fuente} & Administraodr\\  \hline
  \textbf{Estímulo} & Se desea poder tener el control sobre las publicidades del sitio.\\  \hline
  \textbf{Entorno} & Normal\\  \hline
  \textbf{Artefacto} & Sistema en ejecucion\\  \hline
  \textbf{Respuesta} & Se cambian las publicidades que se muestran en las simulaciones.\\  \hline
  \textbf{Medición} & Se modifican solamente 2 repositorios del sistema.\\  \hline
  \end{tabular}
\end{center} 

\begin{center}
  \begin{tabular}{| l | p{10cm} | }
    \hline
  \textbf{Descripción} & La simulaci'on debe poder mejorarse para que sea m'as realista de manera incremental sin que los cambios sean costosos.\\  \hline
  \textbf{Fuente} & Desarrollador\\  \hline
  \textbf{Estímulo} & Se quiere mejorar la simulacion para que sea mas realista.\\  \hline
  \textbf{Entorno} & Normal\\  \hline
  \textbf{Artefacto} & Sistema\\  \hline
  \textbf{Respuesta} & Se mejora el sistema logrando que sea mas realista. \\  \hline
  \textbf{Medición} & Se moficia exactamente un modulo. \\  \hline
  \end{tabular}
\end{center} 



\subsection{Usabilidad}

\begin{center}
  \begin{tabular}{| l | p{10cm} | }
    \hline
  \textbf{Descripción} & Los datos de pago deben guardarse para que el usuario s'olo tenga que actualizarlos espor'adicamente.\\  \hline
  \textbf{Fuente} & Usuario\\  \hline
  \textbf{Estímulo} & Un usuario desea solamente actualizar los datos de pago.\\  \hline
  \textbf{Entorno} & Normal.\\  \hline
  \textbf{Artefacto} & Sistema\\  \hline
  \textbf{Respuesta} & El usuario actualiza los datos de pago.\\  \hline
  \textbf{Medición} & El usuario realiza 4 clicks para actualizar los datos de pago.\\  \hline
  \end{tabular}
\end{center} 


\begin{center}
  \begin{tabular}{| l | p{10cm} | }
    \hline
  \textbf{Descripción} & La interfaz gr'afica de los usuarios debe ser similar a la de un videojuego (sobre todo al momento de ver a los jugadores, las jugadas de los técnicos, colocar el nombre y logo del equipo del participante, con animaciones y efectos especiales con aceleración gráfica). \\  \hline
  \textbf{Fuente} & Externa\\  \hline
  \textbf{Estímulo} & El usuario desea ver la simulacion con una interfaz como la de un videojuego.\\  \hline
  \textbf{Entorno} & Normal\\  \hline
  \textbf{Artefacto} & Sistema\\  \hline
  \textbf{Respuesta} & Se muestra la simulacion como si fuera un videojuego\\  \hline
  \textbf{Medición} & El usuario puede ver la simulacion como si fuera un videojuego haciendo menos de 3 clicks.\\  \hline
  \end{tabular}
\end{center} 



\subsection{Performance}
%TODO
\begin{center}
  \begin{tabular}{| l | p{10cm} | }
    \hline
  \textbf{Descripción} & La calidad del streaming debe adaptarse al bandwidth disponible para que la reproducci'on del video sea fluida.\\  \hline
  \textbf{Fuente} & Interna\\  \hline
  \textbf{Estímulo} & Se desea reproducir un streaming de video.\\  \hline
  \textbf{Entorno} & Normal\\  \hline
  \textbf{Artefacto} & Reproductor de streaming\\  \hline
  \textbf{Respuesta} & Se reproduce el straming en el motor 2D o 3D segun se soporte.\\  \hline
  \textbf{Medición} & En caso de que el bandwidth sea menor que 128Mb/s se usa el motor 2D sino el 3D.\\  \hline
  \end{tabular}
\end{center} 


\begin{center}
  \begin{tabular}{| l | p{10cm} | }
    \hline
  \textbf{Descripción} & Se debe utilizar el engine 3D siempre que sea posible. Si el cliente no lo soporta adecuadamente debe utilizarse el 2D.\\  \hline
  \textbf{Fuente} & Extenera\\  \hline
  \textbf{Estímulo} & Se desea transmitir streaming de video.\\  \hline
  \textbf{Entorno} & Normal\\  \hline
  \textbf{Artefacto} & Cliente\\  \hline
  \textbf{Respuesta} & Dependiendo de las funcionalidades del cliente se muestra el engine 3D o 2D segun corresponda.\\  \hline
  \textbf{Medición} & En el 78\% de los dispositivos clientes se utilizan los motores 3D.\\  \hline
  \end{tabular}
\end{center} 



\begin{center}
  \begin{tabular}{| l | p{10cm} | }
    \hline
  \textbf{Descripción} & Los desf'ios globales que requiere el streaming a m'ultiples regiones simult'aneamente debe funcionar satisfactoriamente. La experiencia del usuario no debe verse afectada.\\  \hline
  \textbf{Fuente} & Externa\\  \hline
  \textbf{Estímulo} & Un usuario visualiza un streaming con multiples regiones.\\  \hline
  \textbf{Entorno} & Normal\\  \hline
  \textbf{Artefacto} & Sistema\\  \hline
  \textbf{Respuesta} & Se procesa el streaming sin demoras ni buffering.\\  \hline
  \textbf{Medición} & En el 0,001\% de los casos el usuario experimenta demoras.\\  \hline
  \end{tabular}
\end{center} 


\subsection{Seguridad}

\begin{center}
  \begin{tabular}{| l | p{10cm} | }
    \hline
  \textbf{Descripción} & Los pagos y las credenciales de pago de los usuarios deben manejarse con seguridad, debe impedirse que hackers redireccionen los pagos o interfieran con las transacciones.\\  \hline
  \textbf{Fuente} & Individuo externo\\  \hline
  \textbf{Estímulo} & Un individuo intenta redireccionar un pago.\\  \hline
  \textbf{Entorno} & Normal\\  \hline
  \textbf{Artefacto} & Sistema\\  \hline
  \textbf{Respuesta} & Se denega el acceso a este usuario y se prohibe que ingrese.\\  \hline
  \textbf{Medición} & En el 0.00001\% de los casos, el individuo externo puedo redireccion el pago.\\  \hline
  \end{tabular}
\end{center} 

\begin{center}
  \begin{tabular}{| l | p{10cm} | }
    \hline
  \textbf{Descripción} & Los datos de usuarios deben estar protegidos contra robos. Tiene que asegurarse la confidencialidad e integridad de los mismos. \\  \hline
  \textbf{Fuente} & Individuo externo\\  \hline
  \textbf{Estímulo} & Un individuo externo intenta acceder a los datos de otros usuarios.\\  \hline
  \textbf{Entorno} & Normal\\  \hline
  \textbf{Artefacto} & Sistema\\  \hline
  \textbf{Respuesta} & Se denega el acceso al individuo externo.\\  \hline
  \textbf{Medición} & En el 0.00001\% de los casos, el individuo externo puedo acceder a los datos de otro usuario.\\  \hline
  \end{tabular}
\end{center} 

\begin{center}
  \begin{tabular}{| l | p{10cm} | }
    \hline
  \textbf{Descripción} & Todas las transacciones de dinero deben estar logueadas de manera segura para poder presentarlas como evidencia a las autoridades de cada regi'on.\\  \hline
  \textbf{Fuente} & Interna\\  \hline
  \textbf{Estímulo} & Se realiza un transaccion de dinero.\\  \hline
  \textbf{Entorno} & Normal\\  \hline
  \textbf{Artefacto} & Sistema\\  \hline
  \textbf{Respuesta} & Se guarda en un log los datos de la transaccion de dinero.\\  \hline
  \textbf{Medición} & El 99.9\% de las transacciones de dinero son guardadas correctamente en el log.\\  \hline
  \end{tabular}
\end{center} 
