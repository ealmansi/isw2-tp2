\documentclass[a4paper, 10pt, twoside]{article}

\usepackage[top=1in, bottom=1in, left=1in, right=1in]{geometry}
\usepackage[utf8]{inputenc}
\usepackage[spanish, es-ucroman, es-noquoting]{babel}
\usepackage{setspace}
\usepackage{fancyhdr}
\usepackage{lastpage}
\usepackage{amsmath}
\usepackage{amsfonts}
\usepackage{amsthm}
\usepackage{verbatim}
\usepackage{fancyvrb}
\usepackage{graphicx}
\usepackage{float}
\usepackage{enumitem} % Provee macro \setlist
\usepackage{tabularx}
\usepackage{multirow}
\usepackage{hyperref}
\usepackage{lscape}
\usepackage{xspace}
\usepackage{qtree}
\usepackage[toc, page]{appendix}


%%%%%%%%%% Constantes - Inicio %%%%%%%%%%
\newcommand{\titulo}{Trabajo Práctico 2}
\newcommand{\materia}{ISW II}
\newcommand{\integrantes}{Almansi · Gasperi · `El capo del sorting` Russo · Tagliavini}
\newcommand{\cuatrimestre}{Primer Cuatrimestre de 2015}
%%%%%%%%%% Constantes - Fin %%%%%%%%%%


%%%%%%%%%% Configuración de Fancyhdr - Inicio %%%%%%%%%%
\pagestyle{fancy}
\thispagestyle{fancy}
\lhead{\titulo\ · \materia}
\rhead{\integrantes}
\renewcommand{\footrulewidth}{0.4pt}
\cfoot{\thepage /\pageref{LastPage}}

\fancypagestyle{caratula} {
   \fancyhf{}
   \cfoot{\thepage /\pageref{LastPage}}
   \renewcommand{\headrulewidth}{0pt}
   \renewcommand{\footrulewidth}{0pt}
}
%%%%%%%%%% Configuración de Fancyhdr - Fin %%%%%%%%%%


%%%%%%%%%% Miscelánea - Inicio %%%%%%%%%%
% Evita que el documento se estire verticalmente para ocupar el espacio vacío
% en cada página.
\raggedbottom

% Separación entre párrafos.
\setlength{\parskip}{0.5em}

% Separación entre elementos de listas.
\setlist{itemsep=0.5em}

% Asigna la traducción de la palabra 'Appendices'.
\renewcommand{\appendixtocname}{Apéndices}
\renewcommand{\appendixpagename}{Apéndices}

\newcommand{\diagrama}[1]{
  \begin{center}
    \includegraphics[width=16cm]{#1}
  \end{center}
}

\newcommand{\diagramadeancho}[2]{
  \begin{center}
    \includegraphics[width=#1]{#2}
  \end{center}
}

\newcommand{\riesgo}[7]{
  \underline{Riesgo {#1}:}
  \begin{itemize}   
    \item \textbf{Descripción:} {#2}
    \item \textbf{Probablidad:} {#3}
    \item \textbf{Impacto:} {#4}
    \item \textbf{Exposición:} {#5}
    \item \textbf{Mitigación:} {#6}
    \item \textbf{Plan de contingencia:} {#7}
  \end{itemize}
}

\newcommand{\escenario}[7] {
  \textit{{#1}}
  \begin{itemize}
    \item \textbf{Fuente:} {#2}
    \item \textbf{Estímulo:} {#3}
    \item \textbf{Entorno:} {#4}
    \item \textbf{Artefacto:} {#5}
    \item \textbf{Respuesta:} {#6}
    \item \textbf{Medición:} {#7}
  \end{itemize}
}

%%%%%%%%%% Miscelánea - Fin %%%%%%%%%%

\begin{document}


%%%%%%%%%%%%%%%%%%%%%%%%%%%%%%%%%%%%%%%%%%%%%%%%%%%%%%%%%%%%%%%%%%%%%%%%%%%%%%%
%% Carátula                                                                  %%
%%%%%%%%%%%%%%%%%%%%%%%%%%%%%%%%%%%%%%%%%%%%%%%%%%%%%%%%%%%%%%%%%%%%%%%%%%%%%%%


\thispagestyle{caratula}

\begin{center}

\includegraphics[height=2cm]{DC.png} 
\hfill
\includegraphics[height=2cm]{UBA.jpg} 

\vspace{2cm}

Departamento de Computación,\\
Facultad de Ciencias Exactas y Naturales,\\
Universidad de Buenos Aires

\vspace{4cm}

\begin{Huge}
\titulo
\end{Huge}

\vspace{0.5cm}

\begin{Large}
\materia
\end{Large}

\vspace{1cm}

\cuatrimestre

\vspace{4cm}

\begin{tabular}{|c|c|c|}
\hline
Apellido y Nombre & LU & E-mail\\
\hline
Almansi, Emilio Guido         & 674/12 & ealmansi@gmail.com\\
Gasperi Jabalera, Fernando    & 56/09  & fgasperijabalera@gmail.com\\
Russo, Christian              & 679/10 & christian.russo8@gmail.com\\
Tagliavini Ponce, Guido       & 783/11 & guido.tag@gmail.com\\
\hline
\end{tabular}

\end{center}

\newpage

\tableofcontents

\newpage


%%%%%%%%%%%%%%%%%%%%%%%%%%%%%%%%%%%%%%%%%%%%%%%%%%%%%%%%%%%%%%%%%%%%%%%%%%%%%%%
%% Introducción                                                              %%
%%%%%%%%%%%%%%%%%%%%%%%%%%%%%%%%%%%%%%%%%%%%%%%%%%%%%%%%%%%%%%%%%%%%%%%%%%%%%%%

\section{Introducción}

Sed sit amet leo sit amet est viverra fermentum sed ac felis. Phasellus ultrices porttitor dolor, quis vehicula felis porta eget. Vestibulum porttitor, quam sit amet pellentesque lobortis, enim velit ullamcorper sapien, eget convallis orci odio id neque. Sed malesuada, diam sit amet hendrerit semper, augue libero sagittis quam, non tincidunt est nulla ut nibh. Donec risus justo, congue sed diam eu, rhoncus sollicitudin nibh. Duis vulputate euismod augue, vel bibendum ex. Sed sed molestie justo. Vivamus ac nibh nec justo bibendum venenatis sed ac mauris. Aliquam a tempus lacus. Vestibulum mollis sapien in fermentum lobortis. Etiam rhoncus, dolor id tristique facilisis, nisi magna iaculis mauris, semper malesuada dolor ante vitae eros. Phasellus sed dui a turpis placerat cursus sed at mauris. Vivamus commodo magna ut ex pharetra volutpat.

Nam urna dolor, semper vel neque sit amet, pellentesque tristique magna. Aenean porta urna ut massa tempus auctor eu sit amet leo. Cras vestibulum ullamcorper nisl id pharetra. Maecenas nec quam in urna pulvinar aliquet. Nulla felis ligula, sodales ullamcorper suscipit ac, mollis at eros. Cras tortor dolor, ornare id ipsum vel, dictum lobortis ipsum. Sed in ante hendrerit, auctor dui sit amet, gravida mauris. Donec magna magna, varius in gravida in, luctus nec massa.


\newpage

%%%%%%%%%%%%%%%%%%%%%%%%%%%%%%%%%%%%%%%%%%%%%%%%%%%%%%%%%%%%%%%%%%%%%%%%%%%%%%%
%% Casos de uso                                                              %%
%%%%%%%%%%%%%%%%%%%%%%%%%%%%%%%%%%%%%%%%%%%%%%%%%%%%%%%%%%%%%%%%%%%%%%%%%%%%%%%

\section{Casos de uso}

Donec non dolor pellentesque, luctus quam in, imperdiet turpis. Pellentesque vehicula sed quam nec imperdiet. Morbi sodales sollicitudin odio ut vehicula. Morbi id blandit arcu, et viverra metus. Cras sed ullamcorper urna. Nulla aliquet orci quam, vel ultrices elit fringilla quis. Etiam tempor venenatis diam, eget vehicula eros scelerisque mattis. Nam sed lorem pretium, commodo lacus a, sodales nunc. Praesent pretium et erat cursus fermentum. Aenean accumsan lectus fermentum sodales dignissim. Nunc at eros id urna mattis condimentum id vitae ligula.

\begin{itemize}

  \item \textbf{Aplicaciones Web/Móvil.}
  \begin{itemize}
    \item Creando nuevo usuario.
    \item Iniciando sesión de usuario.
    \item Ingresando datos de tarjeta / cuenta corriente.
    \item Visualizando desafíos disponibles.
    \item Dando de alta un nuevo desafío.
    \item Ingresando a un desafío como participante.
    \item Armando equipo para desafío.
    \item Visualizando partido de un desafío.
    \item Visualizando saldo de usuario.
    \item Visualizando premios ganados del usuario.
  \end{itemize}

  \item \textbf{Servidor de Juego.}
  \begin{itemize}
    \item Guardando datos de nuevo usuario.
    \item Autenticando datos de usuario.
    \item Guardando datos cifrados de tarjeta / cuenta corriente de usuario.
    \item Transmitiendo un partido de desafío.
    % TODO cambiar esto
    \item Creando desafío.
    \item Cobrando apuestas de desafío.
    \item Cobrando cuota de participación en desafío.
    \item Acreditando ganancias por desafío ganado.
    \item Acreditando premios al usuario.
  \end{itemize}

  \item \textbf{Sistema de Transmisión de Partidos.}
  \begin{itemize}
    \item Transmitiendo datos de una simulación en proceso.
    \item Transmitiendo render de una simulación en proceso.
    \item Transmitiendo un partido en vivo.
  \end{itemize}

  \item \textbf{Sistema de Pagos.}
  \begin{itemize}
    \item Validando datos de tarjeta / cuenta corriente.
    \item Debitando monto a una tarjeta / cuenta corriente.
    \item Acreditando monto a una tarjeta / cuenta corriente.
  \end{itemize}

\end{itemize}

\newpage

%%%%%%%%%%%%%%%%%%%%%%%%%%%%%%%%%%%%%%%%%%%%%%%%%%%%%%%%%%%%%%%%%%%%%%%%%%%%%%%
%% Planificación                                                          %%
%%%%%%%%%%%%%%%%%%%%%%%%%%%%%%%%%%%%%%%%%%%%%%%%%%%%%%%%%%%%%%%%%%%%%%%%%%%%%%%

\section{Planificación}

\subsection{Iteraciones}
Quisque quis ultrices mi. Nunc fringilla velit ut ullamcorper venenatis. Nulla facilisi. Praesent pulvinar suscipit congue. Aliquam finibus eu turpis id pharetra. Nam et magna a purus hendrerit lobortis. Cum sociis natoque penatibus et magnis dis parturient montes, nascetur ridiculus mus. Pellentesque quis enim placerat, euismod odio sed, pretium ligula. Praesent commodo id enim in fringilla. Vivamus dignissim nisl ac diam suscipit tempus. Sed euismod eleifend dolor, in pharetra magna semper sed. Suspendisse elementum id felis in suscipit. Nam ut arcu dignissim, ultricies lorem eu, pretium erat.

\subsubsection{Fase de iniciación}
Sed in blandit nisl. Proin a fermentum metus. Pellentesque imperdiet urna purus, vitae interdum ante luctus at. Integer porttitor eget justo eu molestie. Maecenas nec rhoncus magna. Sed a nunc tempus, condimentum sem ut, volutpat dolor.

\subsubsection{Fase de elaboración}

\textbf{Primera iteración} [n semanas]
\begin{enumerate}
\item Guardando datos de nuevo usuario.
\item Visualizando partido de un desafío.
\item Transmitiendo datos de una simulación en proceso.
\item Creando desafío.
\end{enumerate}

\textbf{Segunda iteración} [n semanas]
\begin{enumerate}
\item Guardando datos cifrados de tarjeta / cuenta corriente de usuario.
\item Autenticando datos de usuario.
\item Ingresando a un desafío como participante.
\item Cobrando apuestas de desafío.
\item Transmitiendo un partido de desafío.
\end{enumerate}

\textbf{Tercera iteración} [n semanas]
\begin{enumerate}
\item Iniciando sesión de usuario.
\item Guardando datos de nuevo usuario.
\item Cobrando cuota de participación en desafío.
\end{enumerate}

\textbf{Cuarta iteración} [n semanas]
\begin{enumerate}
\item Ingresando datos de tarjeta / cuenta corriente.
\item Dando de alta un nuevo desafío.
\item Acreditando premios al usuario.
\item Transmitiendo render de una simulación en proceso.
\end{enumerate}

\textbf{Quinta iteración} [n semanas]
\begin{enumerate}
\item Visualizando desafíos disponibles.
\item Armando equipo para desafío.
\item Visualizando saldo de usuario.
\item Visualizando premios ganados del usuario.
\item Acreditando ganancias por desafío ganado.
\end{enumerate}

\textbf{Sexta iteración} [n semanas]
\begin{enumerate}
\item Transmitiendo un partido en vivo.
\item Validando datos de tarjeta / cuenta corriente.
\item Debitando monto a una tarjeta / cuenta corriente.
\item Acreditando monto a una tarjeta / cuenta corriente.
\end{enumerate}

\subsubsection{Fase de construcción}
Pellentesque in tellus id dui aliquam commodo. Sed aliquet felis mi, sit amet hendrerit elit consectetur vitae. Cras semper felis orci, at suscipit sem sagittis eu. Praesent convallis lectus eu nisi scelerisque, at posuere mauris suscipit. Phasellus ornare, lectus sit amet placerat convallis, metus nulla aliquam ipsum, eu ornare erat tortor eget tortor. Ut ut ultrices elit. Cras porta, neque accumsan scelerisque interdum, magna dolor semper velit, sit amet feugiat nisl mauris et est. Praesent semper quis metus quis sodales.

\subsubsection{Fase de transición}
Vestibulum fermentum posuere sapien, eget congue est mollis ut. Sed ac rutrum purus, non placerat urna. Donec sollicitudin, ex sed auctor eleifend, nibh tortor venenatis metus, a ultricies tellus libero et sapien. Maecenas nec quam et enim euismod placerat et nec libero. Proin vitae mauris imperdiet, tincidunt enim ut, luctus augue. Nunc sed malesuada quam. Curabitur tristique faucibus scelerisque.

\subsection{Alcance de casos de usos de la primera iteración}
\textbf{Guardando datos de nuevo usuario.}

El Servidor de Juego recibió los datos de un usuario a suscribir, y debe almacenarlos en algún medio físico, garantizando su disponibilidad, y haciéndolo de forma segura.

Los datos de un usuario podrían ser almacenados en un único servidor o en un conjunto de servidores. Dependiendo de la escala que pretenda alcanzar la primer versión del sistema, optaremos por un almacenamiento centralizado o distribuído. Si la cantidad de usuarios es del orden de los millones, una base de datos relacional y centralizada puede ser suficiente. Si no, es posible que sea necesaria una base de datos no relacional y distribuída.

La replicación de datos es un factor a considerar, relacionado con la persistencia de estos datos. Observar que la necesidad de replicación depende del volúmen de los datos, sino del volúmen de pedidos al servidor. En efecto, por más que todos los usuarios sean almacenables en forma localizada en una sola máquina, debemos asegurar que estén disponibles, dada la cantidad de accesos a esa información. Para esto podemos replicar una misma base de datos, sobre muchas máquinas.

Con respecto a los datos concretos almacenados, en principio serán sólo los personales, aunque en el futuro también podría utilizarse y guardarse información sobre preferencias y gustos, para la visualización personalizada de publicidad.

Como la seguridad es un atributo de calidad prioritario, debemos almacenar todos estos datos en forma segura, por ejemplo mediante hashing y salt, aunque debe invertirse cierto tiempo en estudiar otras alternativas.

\textbf{Visualizando partido de un desafío.}

La Aplicación Web/Móvil debe mostrar la ejecución de una simulación (si el desafío es de modo simulación) o la transmisión de un partido real (si es de modo fantasía). En la primera iteración, sólo nos interesa visualizar una simulación 2D.

Debe analizarse las tecnologías a utilizar para desarrollar los clientes, dado que para crecer en cantidad de usuarios, necesitamos soportar determinadas plataformas masivamente usadas, como por ejemplo Android.

\textbf{Transmitiendo datos de una simulación en proceso.}

El Servidor de Juego debe enviar a todos los clientes visualizando cierto desafío, el flujo de datos que produce la simulación asociada. Este envío de datos se da únicamente en el caso que un usuario opta por la renderización 2D, y además su conexión de red no soporta la bajada de un flujo de video, que requiere un gran ancho de banda. En estos casos, los datos enviados corresponden a una descripción del transcurso de la simulación, que es renderizada localmente por el usuario.

En la primera iteración definiremos interfaces sencillas para la comunicación entre un cliente y un servidor. El cliente simplemente pedirá los datos de una simulación, y el servidor transmitirá un flujo de datos correspondiente a una simulación artificial.

\textbf{Creando desafío.}

El Servidor de Juego recibió un nuevo desafío, y lo debe almacenar en algún medio físico, garantizando su disponibilidad, y haciéndolo de forma segura.

En esta primera iteración nos interesa decidir cuestiones de arquitectura, disponibilidad y seguridad, similares a las dichas en el alcance del CU \textit{Guardando datos de nuevo usuario}.

\subsection{Tareas CU Primera iteración}
A continuación se detallan las tareas diagramadas para los casos de uso incluidos en la primera iteración con su respectiva estimación de horas hombre.
\\

\begin{tabular}{lp{13cm}l}
  \hline
  CU1 & Guardando datos de nuevo usuario. & 140h \\
  \hline
  T01 & Definir motor de base de datos a utilizar. & 16h \\
  T02 & Definir método de encriptación para contraseñas y datos sensibles. & 16h \\
  T04 & Definir conjunto de datos a persistir por cada usuario. & 16h \\
  T05 & Diseñar base de datos de usuarios. & 16h \\
  T06 & Implementación de CU1 & 40h\\
  T07 & Testing de CU1 & 40h\\
  \hline
\end{tabular}

\vspace{1em}

\begin{tabular}{lp{13cm}l}
  \hline
  CU2 & Visualizando partido de un desafío. & 42h \\
  \hline
  T01 & Investigar tecnologías web a utilizar para el cliente web. & 8h \\
  T02 & Investigar tecnologías a utilizar para los clientes móviles. & 8h \\
  T03 & Desarrollar stubs de las aplicaciones web y móviles. & 8h \\
  T04 & Integrar motor de gráficos 2D a la aplicación web. & 8h \\
  T05 & Integrar motor de gráficos 2D a la aplicación móvil. & 8h \\
  T06 & Implementación de una visualización de partido de prueba. & 8h \\
  T07 & Testing de CU2 & 40h\\
  \hline
\end{tabular}

\vspace{1em}

\begin{tabular}{lp{13cm}l}
  \hline
  CU3 & Transmitiendo datos de una simulación en proceso. & 160h \\
  \hline
  T01 & Investigar plataforma para el servidor de transmisión. & 40h \\
  T02 & Definir la interfaz de comunicación con el servidor. & 40h \\
  T03 & Desarrollar stub del servidor. & 40h \\
  T04 & Implementar transmisión de datos de una simulación de prueba. & 40h \\
  T05 & Desarrollar cliente simple de prueba. & 40h \\
  T06 & Testing de CU3 & 40h\\
  \hline
\end{tabular}

\vspace{1em}

\begin{tabular}{lp{13cm}l}
  \hline
  CU4 & Creando desafío. & 34h \\
  \hline
  T01 & Definir motor de base de datos a utilizar contemplando la escalabilidad. & 16h \\
  T02 & Definir conjunto de datos a persistir por cada desafío. & 16h \\
  T03 & Investigar alternativas para la persistencia segura de datos del desafío. & 4h \\
  T04 & Diseñar base de datos de desafíos. & 16h \\
  T05 & Implementación de CU4 & 40h\\
  T06 & Testing de CU4 & 40h\\
  \hline
\end{tabular}


\subsection{Detalle Primera iteración}

\begin{itemize}
  \item \textbf{Identificación:} E1
  \item \textbf{Tipo de iteración:} Elaboración
  \item \textbf{Cantidad total de horas:} 480
  \item \textbf{Tareas:}
\begin{enumerate}
  \item Definir motor de base de datos a utilizar.
  \item Definir método de encriptación para contraseñas y datos sensibles.
  \item Definir conjunto de datos a persistir por cada usuario.
  \item Diseñar base de datos de usuarios.
  \item Implementación de CU1
  \item Testing de CU1
  \item Investigar tecnologías web a utilizar para el cliente web.
  \item Investigar tecnologías a utilizar para los clientes móviles.
  \item Desarrollar stubs de las aplicaciones web y móviles.
  \item Integrar motor de gráficos 2D a la aplicación web.
  \item Integrar motor de gráficos 2D a la aplicación móvil.
  \item Implementación de una visualización de partido de prueba.
  \item Testing de CU2
  \item Investigar plataforma para el servidor de transmisión.
  \item Definir la interfaz de comunicación con el servidor.
  \item Desarrollar stub del servidor.
  \item Implementar transmisión de datos de una simulación de prueba.
  \item Desarrollar cliente simple de prueba.
  \item Testing de CU3
  \item Definir motor de base de datos a utilizar contemplando la escalabilidad.
  \item Definir conjunto de datos a persistir por cada desafío.
  \item Investigar alternativas para la persistencia segura de datos del desafío.
  \item Diseñar base de datos de desafíos.
  \item Implementación de CU4
  \item Testing de CU4
\end{enumerate}
\end{itemize}

\subsection{Plan de Proyecto}
Donec eu turpis nunc. Cras scelerisque arcu vel arcu commodo euismod. Pellentesque quis urna ac sapien pharetra vulputate ut non nulla. In vitae nisl id nibh varius condimentum. Sed luctus urna viverra leo hendrerit, nec consequat ligula elementum. Etiam lacinia metus in mi dictum, id faucibus mi gravida. Proin vitae massa sagittis, tristique diam vitae, aliquam quam. Mauris tristique turpis diam, rutrum pulvinar magna eleifend vel. Class aptent taciti sociosqu ad litora torquent per conubia nostra, per inceptos himenaeos. Ut sit amet quam et leo pulvinar lobortis. Phasellus a mattis ante. Proin in lorem quis velit facilisis lobortis sit amet eget arcu. Vestibulum pharetra velit nibh.

\begin{landscape}
\begin{figure}[h!]
  \centering
  \includegraphics[width=20cm]{gantt.png}
  \caption{Diagrama de Gantt de la 1era iteración con la asignación de tiempo}
  \label{fig:gantt}
\end{figure}
\end{landscape}
\newpage

%%%%%%%%%%%%%%%%%%%%%%%%%%%%%%%%%%%%%%%%%%%%%%%%%%%%%%%%%%%%%%%%%%%%%%%%%%%%%%%
%% Análisis de riesgo                                                        %%
%%%%%%%%%%%%%%%%%%%%%%%%%%%%%%%%%%%%%%%%%%%%%%%%%%%%%%%%%%%%%%%%%%%%%%%%%%%%%%%

\section{Análisis de riesgos}
\label{riesgos:r1}
\riesgo{1}
    {Etiam sollicitudin sagittis ultrices.}
    {Alta}
    {Alto}
    {Alta}
    {Duis ullamcorper ornare augue, sed pellentesque nulla varius in.}
    {Maecenas a tristique lorem, at mattis dui.}

\riesgo{2}
    {Suspendisse lacinia congue tortor, eu porta ex volutpat consequat.}
    {Media}
    {Medio}
    {Medio}
    {Nulla sed tempor nisl, sit amet eleifend metus.}
    {Ut semper ultricies luctus.}

\riesgo{3}
    {Quisque vitae magna pellentesque, rhoncus erat in, imperdiet odio.}
    {Alta}
    {Alto}
    {Alta}
    {Nunc interdum sagittis sollicitudin.}
    {Praesent blandit pharetra mi non aliquam.}

\newpage

\end{document}
